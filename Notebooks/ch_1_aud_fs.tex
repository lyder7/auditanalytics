% Options for packages loaded elsewhere
\PassOptionsToPackage{unicode}{hyperref}
\PassOptionsToPackage{hyphens}{url}
%
\documentclass[
]{article}
\usepackage{amsmath,amssymb}
\usepackage{lmodern}
\usepackage{iftex}
\ifPDFTeX
  \usepackage[T1]{fontenc}
  \usepackage[utf8]{inputenc}
  \usepackage{textcomp} % provide euro and other symbols
\else % if luatex or xetex
  \usepackage{unicode-math}
  \defaultfontfeatures{Scale=MatchLowercase}
  \defaultfontfeatures[\rmfamily]{Ligatures=TeX,Scale=1}
\fi
% Use upquote if available, for straight quotes in verbatim environments
\IfFileExists{upquote.sty}{\usepackage{upquote}}{}
\IfFileExists{microtype.sty}{% use microtype if available
  \usepackage[]{microtype}
  \UseMicrotypeSet[protrusion]{basicmath} % disable protrusion for tt fonts
}{}
\makeatletter
\@ifundefined{KOMAClassName}{% if non-KOMA class
  \IfFileExists{parskip.sty}{%
    \usepackage{parskip}
  }{% else
    \setlength{\parindent}{0pt}
    \setlength{\parskip}{6pt plus 2pt minus 1pt}}
}{% if KOMA class
  \KOMAoptions{parskip=half}}
\makeatother
\usepackage{xcolor}
\usepackage[margin=1in]{geometry}
\usepackage{color}
\usepackage{fancyvrb}
\newcommand{\VerbBar}{|}
\newcommand{\VERB}{\Verb[commandchars=\\\{\}]}
\DefineVerbatimEnvironment{Highlighting}{Verbatim}{commandchars=\\\{\}}
% Add ',fontsize=\small' for more characters per line
\usepackage{framed}
\definecolor{shadecolor}{RGB}{248,248,248}
\newenvironment{Shaded}{\begin{snugshade}}{\end{snugshade}}
\newcommand{\AlertTok}[1]{\textcolor[rgb]{0.94,0.16,0.16}{#1}}
\newcommand{\AnnotationTok}[1]{\textcolor[rgb]{0.56,0.35,0.01}{\textbf{\textit{#1}}}}
\newcommand{\AttributeTok}[1]{\textcolor[rgb]{0.77,0.63,0.00}{#1}}
\newcommand{\BaseNTok}[1]{\textcolor[rgb]{0.00,0.00,0.81}{#1}}
\newcommand{\BuiltInTok}[1]{#1}
\newcommand{\CharTok}[1]{\textcolor[rgb]{0.31,0.60,0.02}{#1}}
\newcommand{\CommentTok}[1]{\textcolor[rgb]{0.56,0.35,0.01}{\textit{#1}}}
\newcommand{\CommentVarTok}[1]{\textcolor[rgb]{0.56,0.35,0.01}{\textbf{\textit{#1}}}}
\newcommand{\ConstantTok}[1]{\textcolor[rgb]{0.00,0.00,0.00}{#1}}
\newcommand{\ControlFlowTok}[1]{\textcolor[rgb]{0.13,0.29,0.53}{\textbf{#1}}}
\newcommand{\DataTypeTok}[1]{\textcolor[rgb]{0.13,0.29,0.53}{#1}}
\newcommand{\DecValTok}[1]{\textcolor[rgb]{0.00,0.00,0.81}{#1}}
\newcommand{\DocumentationTok}[1]{\textcolor[rgb]{0.56,0.35,0.01}{\textbf{\textit{#1}}}}
\newcommand{\ErrorTok}[1]{\textcolor[rgb]{0.64,0.00,0.00}{\textbf{#1}}}
\newcommand{\ExtensionTok}[1]{#1}
\newcommand{\FloatTok}[1]{\textcolor[rgb]{0.00,0.00,0.81}{#1}}
\newcommand{\FunctionTok}[1]{\textcolor[rgb]{0.00,0.00,0.00}{#1}}
\newcommand{\ImportTok}[1]{#1}
\newcommand{\InformationTok}[1]{\textcolor[rgb]{0.56,0.35,0.01}{\textbf{\textit{#1}}}}
\newcommand{\KeywordTok}[1]{\textcolor[rgb]{0.13,0.29,0.53}{\textbf{#1}}}
\newcommand{\NormalTok}[1]{#1}
\newcommand{\OperatorTok}[1]{\textcolor[rgb]{0.81,0.36,0.00}{\textbf{#1}}}
\newcommand{\OtherTok}[1]{\textcolor[rgb]{0.56,0.35,0.01}{#1}}
\newcommand{\PreprocessorTok}[1]{\textcolor[rgb]{0.56,0.35,0.01}{\textit{#1}}}
\newcommand{\RegionMarkerTok}[1]{#1}
\newcommand{\SpecialCharTok}[1]{\textcolor[rgb]{0.00,0.00,0.00}{#1}}
\newcommand{\SpecialStringTok}[1]{\textcolor[rgb]{0.31,0.60,0.02}{#1}}
\newcommand{\StringTok}[1]{\textcolor[rgb]{0.31,0.60,0.02}{#1}}
\newcommand{\VariableTok}[1]{\textcolor[rgb]{0.00,0.00,0.00}{#1}}
\newcommand{\VerbatimStringTok}[1]{\textcolor[rgb]{0.31,0.60,0.02}{#1}}
\newcommand{\WarningTok}[1]{\textcolor[rgb]{0.56,0.35,0.01}{\textbf{\textit{#1}}}}
\usepackage{graphicx}
\makeatletter
\def\maxwidth{\ifdim\Gin@nat@width>\linewidth\linewidth\else\Gin@nat@width\fi}
\def\maxheight{\ifdim\Gin@nat@height>\textheight\textheight\else\Gin@nat@height\fi}
\makeatother
% Scale images if necessary, so that they will not overflow the page
% margins by default, and it is still possible to overwrite the defaults
% using explicit options in \includegraphics[width, height, ...]{}
\setkeys{Gin}{width=\maxwidth,height=\maxheight,keepaspectratio}
% Set default figure placement to htbp
\makeatletter
\def\fps@figure{htbp}
\makeatother
\setlength{\emergencystretch}{3em} % prevent overfull lines
\providecommand{\tightlist}{%
  \setlength{\itemsep}{0pt}\setlength{\parskip}{0pt}}
\setcounter{secnumdepth}{-\maxdimen} % remove section numbering
\usepackage{float}
\ifLuaTeX
  \usepackage{selnolig}  % disable illegal ligatures
\fi
\IfFileExists{bookmark.sty}{\usepackage{bookmark}}{\usepackage{hyperref}}
\IfFileExists{xurl.sty}{\usepackage{xurl}}{} % add URL line breaks if available
\urlstyle{same} % disable monospaced font for URLs
\hypersetup{
  pdftitle={Fundamentals of Auditing Financial Reports},
  pdfauthor={J. Christopher Westland},
  hidelinks,
  pdfcreator={LaTeX via pandoc}}

\title{Fundamentals of Auditing Financial Reports}
\usepackage{etoolbox}
\makeatletter
\providecommand{\subtitle}[1]{% add subtitle to \maketitle
  \apptocmd{\@title}{\par {\large #1 \par}}{}{}
}
\makeatother
\subtitle{Chapter 1}
\author{J. Christopher Westland}
\date{2023-03-05}

\begin{document}
\maketitle

\hypertarget{auditing}{%
\subsection{Auditing}\label{auditing}}

An audit is an independent examination of the records of an organization
to ascertain how far the financial statements as well as non-financial
disclosures present a true and fair view of the concern. It also
provides assurance that the systems of record keeping are
well-controlled and accurate as required by law. Auditing has become
such a ubiquitous phenomenon in the corporate and the public sector that
academics started identifying an ``Audit Society''.

\hypertarget{r-packages-required-for-this-book}{%
\subsection{R packages required for this
book}\label{r-packages-required-for-this-book}}

The code in the chapters in this book requires R packages that are
specified in the \texttt{library("package\_name")} commands. These will
be packages such as \texttt{tidyverse}, \texttt{ggplot2},
\texttt{lubridate} and \texttt{keras}. Keras is the API for the
Tensorflow machine learning language, and requires a separate
\texttt{keras} installation with \texttt{install\_keras}; general notes
on Tensorflow installation are provided below.

\textbf{Package installation:} There are two steps to using a package.
First it must be \emph{installed}, i.e., copied to a location on your
computer where R can access it. Then it must be \emph{loaded} into the
working memory of \emph{R}. To install, for example the
\texttt{tidyverse} package, type \emph{install.packages(``tidyverse'')}
and then press the \emph{Enter/Return} key. To load the previously
installed package type \emph{library(tidyverse)}. After these commands,
the \texttt{tidyverse} package will now be available for use by your
program code.

\textbf{Tensorflow installation:} Tensorflow is a machine learning
package used in this book; commands to Tensorflow are called using the
Keras API. Prior to using the tensorflow R package you need to install a
version of TensorFlow on your system using the R
\texttt{install\_tensorflow()} function, which provides an easy to use
wrapper for the various steps required to install TensorFlow. You can
also choose to install TensorFlow manually (as described at
\url{https://www.tensorflow.org/install/}).

TensorFlow for R is tested and supported on the following 64-bit
systems:

\begin{enumerate}
\def\labelenumi{\arabic{enumi}.}
\tightlist
\item
  Ubuntu 16.04 or later
\item
  Windows 7 or later
\item
  macOS 10.12.6 (Sierra) or later (no GPU support)
\end{enumerate}

First, install the tensorflow R package from GitHub or the CRAN
respository (search to find the site) then, use the
\texttt{install\_tensorflow()} function to install TensorFlow. Note that
on Windows you need a working installation of Anaconda.
\texttt{install\_tensorflow()} is a wraper around
\texttt{reticulate::py\_install}.

\begin{Shaded}
\begin{Highlighting}[]
\FunctionTok{install.packages}\NormalTok{(}\StringTok{"tensorflow"}\NormalTok{)}
\FunctionTok{library}\NormalTok{(tensorflow)}
\FunctionTok{install\_tensorflow}\NormalTok{()}
\end{Highlighting}
\end{Shaded}

You can confirm that the installation succeeded with:

\begin{Shaded}
\begin{Highlighting}[]
\FunctionTok{library}\NormalTok{(tensorflow)}
\NormalTok{tf}\SpecialCharTok{$}\FunctionTok{constant}\NormalTok{(}\StringTok{"Hellow Tensorflow"}\NormalTok{)}
\DocumentationTok{\#\# tf.Tensor(b\textquotesingle{}Hellow Tensorflow\textquotesingle{}, shape=(), dtype=string)}
\end{Highlighting}
\end{Shaded}

This will provide you with a default installation of TensorFlow suitable
for use with the tensorflow R package. There is much more to Tensorflow,
and interested readers should review the materials at
\url{https://tensorflow.rstudio.com/} and at
\url{https://www.tensorflow.org/}

\end{document}
